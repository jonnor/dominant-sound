% 
% Based on the Template LaTeX for Inter-Noise 2024
\documentclass[a4paper,12pt]{article}

%% Pick the one corresponding to your system
%\usepackage[latin1]{inputenc}
%\usepackage[ansinew]{inputenc}
\usepackage[utf8x]{inputenc}
\usepackage[T1]{fontenc}
\usepackage[colorlinks=true,linkcolor=black,citecolor=black,urlcolor=blue]{hyperref}
\usepackage{geometry}
\geometry{top=2cm,bottom=2cm,left=2.0cm,right=2.0cm}
\pagestyle{empty}


\usepackage{titlesec}
\titleformat{\section}
{\bfseries\uppercase}{\thesection.}{1em}{}
\titleformat{\subsection}
{\bfseries}{\thesection.\thesubsection.}{1em}{}
\renewcommand{\labelitemi}{\textendash}
\renewcommand{\labelitemii}{\textendash}

% Template Packages
\usepackage{graphicx} % used to insert the figure
\usepackage{multirow} % used for the table
\usepackage{cite}
\usepackage{breakurl}
\usepackage{indentfirst}
\usepackage{amsmath, amssymb, amsfonts, bm,stmaryrd}
\usepackage{txfonts}
\usepackage{fourier}
\usepackage{enumitem}
\usepackage{xcolor}
\usepackage{enumitem}

% Our custom packages
\usepackage{todonotes}
\usepackage{algorithm}
\usepackage{algpseudocode}

% Formatting
\hyphenpenalty=10000
\setlength{\emergencystretch}{3em}

\columnsep 1cm
\setlength{\parindent}{1.0cm}
\setlength{\parskip}{0.0cm}

\titlespacing*{\subsection}{0pt}{1.5em}{0.2em}
\titlespacing*{\section}{0pt}{1.5em}{0.2em}

\renewcommand\eqref[1]{Equation~\ref{#1}}

\renewcommand{\thesection}{\arabic{section}}
\renewcommand{\thesubsection}{\arabic{subsection}}
\renewcommand{\arraystretch}{1.25}

\makeatletter
\renewcommand\@biblabel[1]{#1.}
\makeatother

\renewcommand{\refname}{REFERENCES} 
\setlength{\footnotesep}{12pt} 
% Some stuff with item spacing in lists, related to enumitem package.   See individual lists. 
%\setlist[2]{noitemsep}
%\setenumerate{noitemsep}

% Sort out linespacing with references
\newlength{\bibitemsep}\setlength{\bibitemsep}{.2\baselineskip plus .05\baselineskip minus .05\baselineskip}
\newlength{\bibparskip}\setlength{\bibparskip}{0pt}
\let\oldthebibliography\thebibliography
\renewcommand\thebibliography[1]{%
  \oldthebibliography{#1}%
  \setlength{\parskip}{\bibitemsep}%
  \setlength{\itemsep}{\bibparskip}%
}

%===============================================================================
% You should only need to change this section to create a template for a new conference
\newcommand{\YearConf}{2024}
\newcommand{\CityConf}{NANTES}
\newcommand{\CityConfa}{Nantes}
\newcommand{\DateConf}{25-28 August \YearConf}
\newcommand{\CountryConf}{France}
\newcommand{\LogoConf}{logo_IN24.jpg}
\newcommand{\CopyrightConf}{Permission is granted for the reproduction of a fractional part of this paper published in the Proceedings of INTER-NOISE \YearConf ~ \underline{provided permission is obtained} from the author(s) \underline{and credit is given} to the author(s) and these proceedings.}
%==============================================================================


%==============================================================================
\usepackage{fancyhdr}
\pagestyle{fancy}
\fancyhead{}\fancyfoot{}
\renewcommand\headrulewidth{1pt}
\chead{Proceedings of INTER-NOISE \YearConf}
  \fancypagestyle{firststyle}
{   \fancyhf{}
   \fancyfoot[C]{\scriptsize \CopyrightConf}
   \renewcommand{\headrulewidth}{0pt} % removes horizontal header line
}

% DOCUMENT BEGIN
%==============================================================
\begin{document}
\thispagestyle{firststyle}

\begin{center}
	\includegraphics[width=2in]{\LogoConf}
\end{center}
\vskip.5cm

\begin{flushleft}
\fontsize{16}{20}\selectfont\bfseries
The Dominant Noise Event method for automatic classification in noise monitoring
\end{flushleft}
\vskip1cm

\renewcommand\baselinestretch{1}
\begin{flushleft}
Jon Nordby\footnote{jon@soundsensing.no}\\
Soundsensing\\
Oslo, Norway\\

\vskip.5cm
Blake Downward\footnote{blake@foxlimbry.com}\\
Fox Limbry\\
Adelaide, Australia\\

\end{flushleft}
\textbf{\centerline{ABSTRACT}}\\
\textit{We propose a method for noise monitoring that can automatically classify noise 
events, and determine the sound level contribution from different noise sources. We observe that in some noise monitoring scenarios, the main phenomena of interest are intrusive noise events, which are considerably louder than the background.
When the noise events dominate the background level, it is possible to assign the sound level contribution at each time-period to a single event class.
Our contributions are as follows:
A) a formalization of two Machine Learning tasks designed to consider the sound level contribution during classification, tentatively called: Dominant Sound Classification and Dominant Sound Event Detection.
B) an analysis of to which degree the dominant sound simplification holds in noise monitoring scenarios
C) a demonstration of a complete system that attributes sound level contributions to noise source classes
The method is evaluated quantitatively using representative real-world datasets.
}
\section{INTRODUCTION}
\noindent
% Disposition: 0.5 page. Ideally fit on frontpage
\todo[inline]{motivation, relevance, context}

The availability and cost of Noise Monitoring systems that can measure soundlevels
over longer periods of time are decreasing steadily.
Recent research has also shown that it is possible to use Machine Learning 
to automatically classify audio clips and sound events.
However there is still a lack of tools and methodologies that combine the results
of high-resolution soundlevel monitoring with automated classification,
in order to create automated noise monitoring systems that can quantify
how different sound sources contribute to the overall noise impact.

% The length of a manuscript should be at most 12 pages and at least 4 pages. 

\clearpage
\section{Background}
% Disposition: 0.5-1 page

\todo[inline]{background. Related works. Prior art. Fundamental concepts, key theory, used in methods etc.}

ML task definitions. Audio Classification, Sound Event Detection

Standard/practices for Noise Monitoring with sound levels, limits etc.

Existing works using ML for Noise Monitoring.

\clearpage
\section{Methods}
% Disposition: 1 page
\todo[inline]{methods. How do we structure the experiments }

\floatname{algorithm}{Task}
\renewcommand{\algorithmicrequire}{\textbf{Input:}}
\renewcommand{\algorithmicensure}{\textbf{Output:}}

TLDR: Modification on (closed-set) Audio Classification,
where the output class shall be the one that contributes the most to the soundlevel.
\begin{algorithm}
\caption{Dominant Sound Classification}\label{alg:cap}
\begin{algorithmic}
\Require Stream of audio.
Consisting of sound from multiple different sound sources.
\Ensure Single class label.
Being the class that contributed the most to the soundlevel.
\end{algorithmic}
\end{algorithm}

TLDR: Modification on (monophonic) Sound Event Detection,
where the output class shall be the one that contributes the most to the soundlevel.

\begin{algorithm}
\caption{Dominant Sound Event Detection}\label{alg:cap}
\begin{algorithmic}
\Require Stream of audio.
Consisting of sound events from multiple different sound sources.
\Ensure A stream of class activations over time one per point in time.
The active class shall be the one that made the most contribution to the soundlevel.
High resolution at output, corresponding with soundlevel measurement. For example LA_fast (125 ms) or LA_slow (1000 ms).
\end{algorithmic}
\end{algorithm}

In addition to `{class0,class1....classN}`, also allow for two pseudo-classes: {unknown,mixture}. `unknown` means the model cannot reliably detect the class.
`mixture` means that the dominant-source assumption is violated,
there are multiple sources that significantly contribute to the overall noise level.

\clearpage
\section{Results}
\noindent
% Disposition: 1 page
\todo[inline]{results. Including tables and plots}


\section{Conclusions}
\noindent
% Disposition. 0.25 page
\todo[inline]{conclusions}

\section*{Acknowledgements}
\noindent
% Disposition. 0 to 4 sentences
\todo[inline]{acknowledgements (if any). Funding sources, advisors, etc?}


%% ---- Instructions etc ---- TO BE DELETED %
\clearpage
\section{Formatting and instructions}
\todo[inline]{DELETE WHEN NO LONGER USED}

\subsection{Figures, Equations, Tables}
\noindent
All figures, tables, equations, photos, graphs, etc., must be shown shortly after they are mentioned, placed at the centre of a page as shown in Figure \ref{fig:1} below.

The captions of figures and photos are put below the figures and photos (see Figure \ref{fig:1}).  They are centered if one line or less long, and fully justified if longer than one line.  They should be referred to in the text as Figure 1, Figure 2, etc. 
\begin{figure}[h!]
\begin{center}
  \includegraphics[width=2in]{\LogoConf}
  \end{center}
  \caption{The INTER-NOISE \YearConf   ~logo.}
  \label{fig:1}
\end{figure}

The equations should be referenced as Equation 1, Equation 2, etc. For example: the formula for estimating the mean value, is:

\begin{equation}
\bar{X} = \frac{1}{N} \sum_{i=1}^{N} X_i ,
\label{Eq:1}
\end{equation}
\noindent
where $X_i$ are the measurements, and $N$ is the number of measurements  The equation should be part of the sentence, either with a period at the end, or a comma as in Equation \ref{Eq:1} shown above where after the displayed equation, the variables are explained as part of the sentence that contains the displayed equation. 

The caption for a table should be placed just above the table and the table number should be Table 1, Table 2,  etc. like Table \ref{Tab:1} below.  The caption is centered if one line long or less, and fully justified if longer than one line.  Tables should be referred to in the text as, e.g., Table \ref{Tab:1}.

\begin{table}[h!]

\caption{Example of values displayed in a table. The header row is bold, and the columns are not usually separated by lines.}
\label{Tab:1}

\begin{center}
\begin{tabular}{c c c } 
 \hline
 \textbf{Test Number} &  \textbf{Variable 1}& \textbf{Variable 2}  \\ [0.5ex] 
 \hline
 1 & 6.1 & a \\ 
 \hline
 2 & 7.2 & b  \\
 \hline
 3 & 3.3 & c \\ [1ex] 
 \hline
\end{tabular}
\end{center}

\end{table}


\section{Referencing other work}
\noindent

\cite{latexcompanion, knuthwebsite}


\section{IMPORTANT Uploading INFORMATION}
   \label{important}
\noindent
Here are the instructions for submitting manuscripts, with some comments on conversion to PDF.
\subsection{Submission of Manuscripts}
\noindent
Submit your manuscript as a PDF file using the link on the INTER-NOISE \YearConf ~ website (www.internoise\YearConf.org).  See subsection \ref{important}.\ref{conversion} about checking your conversion to PDF.
The file name for your paper should be of the form:  IN\_\YearConf\_XXXX.pdf, where XXXX is your abstract number. 

\subsection{Conversion to PDF}
  \label{conversion}
\noindent
Before submission, you need to check your PDF file carefully to be sure that PDF conversion was done properly and there is no error when the PDF file is opened. The following problems may occur.
Compare the formatting to the InterNoise 2024 Template.PDF file.
\begin{itemize}[noitemsep]
\item
Symbols are missed.
\item
Symbols are converted incorrectly, especially mathematical symbols.
\item
Figures are missed.
\item
Indentation is not correct.
\end{itemize}



\bibliographystyle{unsrt}
\bibliography{sample} 

\end{document}

